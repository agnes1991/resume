\documentclass[11pt,a4paper]{moderncv}

% moderncv themes
%\moderncvtheme[blue]{casual}                 % optional argument are 'blue' (default), 'orange', 'red', 'green', 'grey' and 'roman' (for roman fonts, instead of sans serif fonts)
\moderncvtheme[orange]{classic}                % idem
\usepackage{xunicode, xltxtra}
\XeTeXlinebreaklocale "zh"
\widowpenalty=10000

%\setmainfont[Mapping=tex-text]{微软雅黑}
%\setmainfont[Mapping=tex-text]{文泉驿正黑}

% character encoding
%\usepackage[utf8]{inputenc}                   % replace by the encoding you are using
\usepackage{CJKutf8}

% adjust the page margins
\usepackage[scale=0.8]{geometry}
\recomputelengths                             % required when changes are made to page layout lengths
% \setmainfont[Mapping=tex-text]{微软雅黑}
% \setsansfont[Mapping=tex-text]{微软雅黑}
\setmainfont[Mapping=tex-text]{Hiragino Sans GB}
\setsansfont[Mapping=tex-text]{Hiragino Sans GB}
\CJKtilde

% personal data

%% start of file `template-zh.tex'.
%% Copyright 2006-2012 Xavier Danaux (xdanaux@gmail.com).
%
% This work may be distributed and/or modified under the
% conditions of the LaTeX Project Public License version 1.3c,
% available at http://www.latex-project.org/lppl/.

% 个人信息
\firstname{黄}
\familyname{瑞}
\title{个人简历}                      % 可选项、如不需要可删除本行
\address{深圳市}{广东省}             % 可选项、如不需要可删除本行
\mobile{+86~15994801253}                         % 可选项、如不需要可删除本行
%\phone{+2~(345)~678~901}                          % 可选项、如不需要可删除本行
%\fax{+3~(456)~789~012}                            % 可选项、如不需要可删除本行
\email{rui\_huang@outlook.com}                    % 可选项、如不需要可删除本行
% \homepage{duguying.net}                  % 可选项、如不需要可删除本行
\extrainfo{26岁、女}                  % 可选项、如不需要可删除本行
% \photo[64pt]{avatar.jpg}                  % ‘64pt’是图片必须压缩至的高度、‘0.4pt‘是图片边框的宽度 (如不需要可调节至0pt)、’picture‘ 是图片文件的名字;可选项、如不需要可删除本行
%\quote{引言(可选项)}                           % 可选项、如不需要可删除本行

% 显示索引号;仅用于在简历中使用了引言
%\makeatletter
%\renewcommand*{\bibliographyitemlabel}{\@biblabel{\arabic{enumiv}}}
%\makeatother

% 分类索引
%\usepackage{multibib}
%\newcites{book,misc}{{Books},{Others}}
%----------------------------------------------------------------------------------
%            内容
%----------------------------------------------------------------------------------
\begin{document}
\maketitle

\section{教育背景}
\cventry{2010 -- 2014}{本科}{长江大学}{计算机科学与技术学院}{计算机科学与技术专业}{}  % 第3到第6编码可留白

%\section{毕业论文}
%\cvitem{题目}{\emph{题目}}
%\cvitem{导师}{导师}
%\cvitem{说明}{\small 论文简介}

% \section{求职意向}
% \cvitem{工作性质}{全职}
% \cvitem{期望职业}{软件测试}
% \cvitem{期望行业}{IT服务(系统/数据/维护)}
% \cvitem{工作地区}{深圳}
% \cvitem{期望月薪}{8001-10000元/月}
% \cvitem{目前状况}{我目前在职,正考虑换个新环境(如有合适的工作机会,到岗时间一个月左右)}

\section{工作经历}
\renewcommand{\baselinestretch}{1.2}

\cventry{2016/11\\至今}
{雷鸟}{TCL}{}{}
\cventry{2016/02\\2016/11}
{深圳市奥思网络科技有限公司(开源中国)}{}{}{}
\cventry{2014/07\\2015/12}
{武汉百钧成科技有限公司}{}{}{}
\cventry{2013/04\\2014/05}
{微软中国有限公司}}{}{}{}

\section{项目经历}

{性能测试}{}{}{}
{责任描述:公司所有涉及后台API接口的功能及性能测试工作\\
1. 性能测试。通过ab命令及locust脚本对项目相关接口进行性能测试,包括使用python编写性能测试脚本等\\
2. 编写测试文档、用户文档等\\
}

{雷鸟官网测试}{}{}{}
{责任描述:雷鸟官网封闭式开发项目测试工作\\
1. 模块功能测试。基础功能测试\\
2. 安全测试。XSS、CSRF、权限等安全测试\\
3. 性能测试。通过ab命令及locust脚本对项目相关接口进行性能测试\\
4. 编写测试文档、用户文档等\\
5. 兼容性测试。对各个浏览器平台(web端和移动端)进行兼容性测试
}

\cventry{2017/02\\2017/03}
{智屏广告}{}{}{}
{责任描述:智屏广告项目测试工作:\\
1. 模块功能测试。基础功能测试\\
2. 安全测试。XSS、CSRF、权限等安全测试\\
3. 编写测试文档、用户文档等
}

\cventry{2016/03\\2016/11}
{开源中国众包产品}{}{}{}
{责任描述:开源中国众包产品测试工作:\\
1. 模块功能测试。基础功能测试\\
2. 安全测试。XSS、CSRF、权限等安全测试\\
3. 自动化测试。编写自动化测试用例\\
4. 编写测试文档、用户文档等
}

\cventry{2015/06\\至\\2016/01}
{eSupplier供应商协调系统}{}{}{}
{责任描述:eSupplier项目采用敏捷模式开发,2周一个迭代,每2个月交付一次版本。在项目中负责系统主流程中的验收模块以及终端模块的测试工作:\\
1. 与需求分析人员沟通需求\\
2. 编写测试用例\\
3. 根据用例实施测试。包含功能测试、易用性测试机兼容性测试\\
4. 协助测试团队其他成员进行测试工作\\
5. 输出测试报告\\
6. 编写编写所负责模块的集成测试文档、业务梳理文档、用户手册等文档
}

% \cvlistitem{编写编写所负责模块的集成测试文档、业务梳理文档、用户手册等文档}

\cventry{2014/12\\至\\2015/05}
{Cookies取信系统}
{}
{}{}
{软件环境:Tomcat、Java、Apache\\
硬件环境:服务器、客户端\\
开发工具:Visual Studio,MySQL\\
责任描述:负责项目系统测试,系统稳定编写,测试文档编写等工作
}

\cventry{2014/07\\至\\2015/05}
{邮件综合管理平台}
{}
{}{}
{软件环境:Tomcat、Java、Apache、SQL Server\\
硬件环境:服务器、客户端\\
开发工具:Visual Studio\\
责任描述:负责项目系统测试,测试全部文档编写,系统部分文档编写,系统版本发布管理,测试服务器管理,系统运行维护等工作。
}

\cventry{2013/11\\至\\2014/05}
{中国石化基础设施云}
{}
{}{}
{责任描述:负责项目测试。系统测试计划、测试用例编写,系统初步测试,bug汇报反馈等。协助建设资源池机房,安装服务器系统、配置服务器磁盘阵列\\
项目简介:中国石化未来的基础设施建设需要实现统一建设、统一管理、资源整合、自动化运维、灵活调整资源等,提高基础设施整体的可用性、利用率。因此使用虚拟化技术(Hyper-v)与System Center 2012 SP1产品相结合的部署架构实现对基础设施建设提供有效的解决方案。}

\cventry{2013/04\\至\\2014/04}
{中国石化统一通讯系统}
{}
{}{}
{责任描述:负责系统的测试工作。主要负责黑盒测试,从输入/输出的数据对系统是否出现不该存在的问题进行判断;兼容性测试,分别将项目部署在多个不同的系统环境中进行测试,以判断系统在不同环境中运行时是否会出现异常;回归测试,当需要发布新的系统版本是对系统进行回归测试;性能测试,使用工具对系统进行检测,以判断是否能在20w用户的使用下正常工作\\
项目简介:中国石化统一通讯系统是以Lync2010为基础,构建中国石化的统一通讯平台,实现可集中管控的沟通服务资源,满足通讯自身的灵活性、集成性、统一性。统一通讯系统功能包括用户功能、管理功能以及集成功能。用户功能包括即使消息、语音、视频、网络会议、应用扩展、企业通讯录、群组等功能,提高用户沟通的效率,由基于Lync标准客户端的开发来实现。
}

\section{校园经历}
\renewcommand{\baselinestretch}{1.2}

\cventry{2011–2012}
{长江大学计算机协会}
{}
{}{}
{担任协会总会长。负责协会全部事物,包括会员招纳,会员活动组织 ,协会理事会管理等工作。期间获得社团优秀干部、先进个人等称号。}

\cventry{2011–2012}
{长江大学计算机学院学生会}
{}
{}{}
{担任生活部部长,协助学院辅导员进行学院学生的各项考勤及安全事务管理。期间获得五四表彰“优秀团干”称号。}

\cventry{2011–2012}
{长江大学计算机科学学院辅导员助理}
{}
{}{}
{协助院辅导员处理各项学生事务。}



\section{专业技能}
\cvlistitem{自动化测试、移动端测试、性能测试、接口测试、功能测试、安全测试、易用性测试、兼容性测试}
\cvlistitem{掌握Python+Selenium以及Java+Selenium自动化测试用例输出}
\cvlistitem{熟练编写测试计划、测试用例、测试报告等文档}
\cvlistitem{掌握Markdown语法}
\cvlistitem{掌握Git}
\cvlistitem{熟练使用TFS、TMSS、SVN等工具}
\cvlistitem{用户培训,可以为用户进行不同程度的系统培训}

\section{个人技能}
\cvline{语言技能}{英语\textbf{CET-6},能够熟练阅读并书写文档,并可以与人进行交流}
\cvline{其他技能}{羽毛球}


% \section{自我评价}
% \cvitem{}{工作态度认真负责,会认真完成自己的工作,并努力将工作做到更好。沟通能力强,与同事或客户之间可以流畅的沟通,并且具备英文交谈能力。学习能力和理解能力强,可以很快的接受并学会新的事物或技能。抗压能力强,可以适应高强度的工作节奏。思维清晰逻辑连贯,细致认真,懂得反省自身存在的问题并努力加以改进,是我最大的优点。}


\closesection{}                   % needed to renewcommands
\renewcommand{\listitemsymbol}{-} % change the symbol for lists

% 来自BibTeX文件但不使用multibib包的出版物
%\renewcommand*{\bibliographyitemlabel}{\@biblabel{\arabic{enumiv}}}% BibTeX的数字标签
\nocite{*}
\bibliographystyle{plain}
\bibliography{publications}                    % 'publications' 是BibTeX文件的文件名

% 来自BibTeX文件并使用multibib包的出版物
%\section{出版物}
%\nocitebook{book1,book2}
%\bibliographystylebook{plain}
%\bibliographybook{publications}               % 'publications' 是BibTeX文件的文件名
%\nocitemisc{misc1,misc2,misc3}
%\bibliographystylemisc{plain}
%\bibliographymisc{publications}               % 'publications' 是BibTeX文件的文件名

\end{document}


%% 文件结尾 `template-zh.tex'.
